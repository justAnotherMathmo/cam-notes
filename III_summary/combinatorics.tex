\documentclass[a4paper]{article}

\def\npart {III}
\def\nterm {Michaelmas}
\def\nyear {2016}
\def\nlecturer {B. Bollob\'as}
\def\ncourse {Combinatorics}

% Imports
\author{Based on lectures by \nlecturer \\\small Summary created by Kaimyn Chapman-Brown \\\small Framework created by Dexter Chua}
\date{\nterm\ \nyear}

\usepackage{alltt}
\usepackage{amsfonts}
\usepackage{amsmath}
\usepackage{amssymb}
\usepackage{amsthm}
\usepackage{booktabs}
\usepackage{caption}
\usepackage{enumitem}
\usepackage{fancyhdr}
\usepackage{graphicx}
\usepackage{mathdots}
\usepackage{mathrsfs}
\usepackage{marginnote}
\usepackage{mathtools}
\usepackage{microtype}
\usepackage{multirow}
\usepackage{pdflscape}
\usepackage{pgfplots}
% % \usepackage{siunitx}
\usepackage{tabularx}
\usepackage{tikz}
\usepackage{tkz-euclide}
\usepackage[normalem]{ulem}
\usepackage[all]{xy}
\usepackage{imakeidx}

\makeindex[intoc, title=Index]
\indexsetup{othercode={\lhead{\emph{Index}}}}

\ifx \nextra \undefined
  \usepackage[pdftex,
    hidelinks,
    pdfauthor={Dexter Chua},
    pdfsubject={Cambridge Maths Notes: Part \npart\ - \ncourse},
    pdftitle={Part \npart\ - \ncourse},
  pdfkeywords={Cambridge Mathematics Maths Math \npart\ \nterm\ \nyear\ \ncourse}]{hyperref}
  \title{Part \npart\ --- \ncourse}
\else
  \usepackage[pdftex,
    hidelinks,
    pdfauthor={Dexter Chua},
    pdfsubject={Cambridge Maths Notes: Part \npart\ - \ncourse\ (\nextra)},
    pdftitle={Part \npart\ - \ncourse\ (\nextra)},
  pdfkeywords={Cambridge Mathematics Maths Math \npart\ \nterm\ \nyear\ \ncourse\ \nextra}]{hyperref}

  \title{Part \npart\ --- \ncourse \\ {\Large \nextra}}
  \renewcommand\printindex{}
\fi

\pgfplotsset{compat=1.12}

\pagestyle{fancyplain}
\lhead{\emph{\nouppercase{\leftmark}}}
\ifx \nextra \undefined
  \rhead{
    \ifnum\thepage=1
    \else
      \npart\ \ncourse
    \fi}
\else
  \rhead{
    \ifnum\thepage=1
    \else
      \npart\ \ncourse\ (\nextra)
    \fi}
\fi
\usetikzlibrary{arrows}
\usetikzlibrary{decorations.markings}
\usetikzlibrary{decorations.pathmorphing}
\usetikzlibrary{positioning}
\usetikzlibrary{fadings}
\usetikzlibrary{intersections}
\usetikzlibrary{cd}

\newcommand*{\Cdot}{\raisebox{-0.25ex}{\scalebox{1.5}{$\cdot$}}}
\newcommand {\pd}[2][ ]{
  \ifx #1 { }
    \frac{\partial}{\partial #2}
  \else
    \frac{\partial^{#1}}{\partial #2^{#1}}
  \fi
}
\ifx \nhtml \undefined
\else
  \renewcommand\printindex{}
  \makeatletter
  \DisableLigatures[f]{family = *}
  \let\Contentsline\contentsline
  \renewcommand\contentsline[3]{\Contentsline{#1}{#2}{}}
  \renewcommand{\@dotsep}{10000}
  \newlength\currentparindent
  \setlength\currentparindent\parindent

  \newcommand\@minipagerestore{\setlength{\parindent}{\currentparindent}}
  \usepackage[active,tightpage,pdftex]{preview}
  \renewcommand{\PreviewBorder}{0.1cm}

  \newenvironment{stretchpage}%
  {\begin{preview}\begin{minipage}{\hsize}}%
    {\end{minipage}\end{preview}}
  \AtBeginDocument{\begin{stretchpage}}
  \AtEndDocument{\end{stretchpage}}

  \newcommand{\@@newpage}{\end{stretchpage}\begin{stretchpage}}

  \let\@real@section\section
  \renewcommand{\section}{\@@newpage\@real@section}
  \let\@real@subsection\subsection
  \renewcommand{\subsection}{\@@newpage\@real@subsection}
  \makeatother
\fi

% Theorems
\theoremstyle{definition}
\newtheorem*{aim}{Aim}
\newtheorem*{axiom}{Axiom}
\newtheorem*{claim}{Claim}
\newtheorem*{conj}{Conjecture}
\newtheorem*{cor}{Corollary}
\newtheorem*{defi}{Definition}
\newtheorem*{eg}{Example}
\newtheorem*{fact}{Fact}
\newtheorem*{law}{Law}
\newtheorem*{lemma}{Lemma}
\newtheorem*{notation}{Notation}
\newtheorem*{prop}{Proposition}
\newtheorem*{rem}{Remarks}
\newtheorem*{thm}{Theorem}

\newtheorem{cor-num}{Corollary}
\newtheorem{defi-num}[cor-num]{Definition}
\newtheorem{exer-num}[cor-num]{Exercise}
\newtheorem{prop-num}[cor-num]{Proposition}
\newtheorem{thm-num}[cor-num]{Theorem}


\renewcommand{\labelitemi}{--}
\renewcommand{\labelitemii}{$\circ$}
\renewcommand{\labelenumi}{(\roman{*})}

\let\stdsection\section
\renewcommand\section{\newpage\stdsection}

% Strike through
\def\st{\bgroup \ULdepth=-.55ex \ULset}

% Maths symbols
\newcommand{\bra}{\langle}
\newcommand{\ket}{\rangle}

\newcommand{\N}{\mathbb{N}}
\newcommand{\Z}{\mathbb{Z}}
\newcommand{\Q}{\mathbb{Q}}
%\renewcommand{\H}{\mathbb{H}}
\newcommand{\R}{\mathbb{R}}
\newcommand{\C}{\mathbb{C}}
\newcommand{\Prob}{\mathbb{P}}
\renewcommand{\P}{\mathbb{P}}
\newcommand{\E}{\mathbb{E}}
\newcommand{\F}{\mathbb{F}}
\newcommand{\cU}{\mathcal{U}}
\newcommand{\RP}{\mathbb{RP}}
\newcommand{\CP}{\mathbb{CP}}

\newcommand{\A}{\mathscr{A}}	%Backwards compatability
\newcommand{\calA}{\mathscr{A}}
\newcommand{\calB}{\mathscr{B}}
\newcommand{\calC}{\mathscr{C}}
\newcommand{\calD}{\mathscr{D}}
\newcommand{\calO}{\mathscr{O}}


\newcommand{\ph}{\,\cdot\,}
\def\lc{\left\lceil}   
\def\rc{\right\rceil}
\def\lf{\left\lfloor}   
\def\rf{\right\rfloor}


\let\Im\relax
\let\Re\relax
\DeclareMathOperator{\adj}{adj}
\DeclareMathOperator{\Ann}{Ann}
\DeclareMathOperator{\area}{area}
\DeclareMathOperator{\Aut}{Aut}
\DeclareMathOperator{\Bernoulli}{Bernoulli}
\DeclareMathOperator{\betaD}{beta}
\DeclareMathOperator{\bias}{bias}
\DeclareMathOperator{\binomial}{binomial}
\DeclareMathOperator{\card}{card}
\DeclareMathOperator{\ccl}{ccl}
\DeclareMathOperator{\Char}{char}
\DeclareMathOperator{\cl}{cl}
\DeclareMathOperator{\corr}{corr}
\DeclareMathOperator{\cosec}{cosec}
\DeclareMathOperator{\cosech}{cosech}
\DeclareMathOperator{\cov}{cov}
\DeclareMathOperator{\covol}{covol}
\DeclareMathOperator{\diag}{diag}
\DeclareMathOperator{\disc}{disc}
\DeclareMathOperator{\dom}{dom}
\DeclareMathOperator{\End}{End}
\DeclareMathOperator{\energy}{energy}
\DeclareMathOperator{\erfc}{erfc}
\DeclareMathOperator{\erf}{erf}
\DeclareMathOperator{\ev}{ev}
\DeclareMathOperator{\Fit}{Fit}
\DeclareMathOperator{\fix}{fix}
\DeclareMathOperator{\Frac}{Frac}
\DeclareMathOperator{\Gal}{Gal}
\DeclareMathOperator{\gammaD}{gamma}
\DeclareMathOperator{\hcf}{hcf}
\DeclareMathOperator{\Hom}{Hom}
\DeclareMathOperator{\id}{id}
\DeclareMathOperator{\image}{image}
\DeclareMathOperator{\im}{im}
\DeclareMathOperator{\Im}{Im}
\DeclareMathOperator{\Ind}{Ind}
\DeclareMathOperator{\Int}{Int}
\DeclareMathOperator{\Isom}{Isom}
\DeclareMathOperator{\lcm}{lcm}
\DeclareMathOperator{\length}{length}
\DeclareMathOperator{\like}{like}
\DeclareMathOperator{\Lk}{Lk}
\DeclareMathOperator{\Mat}{Mat}
\DeclareMathOperator{\mse}{mse}
\DeclareMathOperator{\multinomial}{multinomial}
\DeclareMathOperator{\orb}{orb}
\DeclareMathOperator{\ord}{ord}
\DeclareMathOperator{\otp}{otp}
\DeclareMathOperator{\Poisson}{Poisson}
\DeclareMathOperator{\rank}{rank}
\DeclareMathOperator{\rel}{rel}
\DeclareMathOperator{\Re}{Re}
\DeclareMathOperator*{\res}{res}
\DeclareMathOperator{\Res}{Res}
\DeclareMathOperator{\Root}{Root}
\DeclareMathOperator{\sech}{sech}
\DeclareMathOperator{\sgn}{sgn}
\DeclareMathOperator{\spn}{span}
\DeclareMathOperator{\stab}{stab}
\DeclareMathOperator{\St}{St}
\DeclareMathOperator{\supp}{supp}
\DeclareMathOperator{\Syl}{Syl}
\DeclareMathOperator{\Sym}{Sym}
\DeclareMathOperator{\tr}{tr}
\DeclareMathOperator{\var}{var}
\DeclareMathOperator{\vol}{vol}

\newcommand{\GL}{\mathrm{GL}}
\newcommand{\SL}{\mathrm{SL}}
\newcommand{\PGL}{\mathrm{PGL}}
\newcommand{\PSL}{\mathrm{PSL}}
\newcommand{\PSU}{\mathrm{PSU}}
\newcommand{\Or}{\mathrm{O}}
\newcommand{\SO}{\mathrm{SO}}
\newcommand{\U}{\mathrm{U}}
\newcommand{\SU}{\mathrm{SU}}

\renewcommand{\d}{\mathrm{d}}
\newcommand{\D}{\mathrm{D}}

\pgfarrowsdeclarecombine{twolatex'}{twolatex'}{latex'}{latex'}{latex'}{latex'}
\tikzset{->/.style = {decoration={markings,
                                  mark=at position 1 with {\arrow[scale=2]{latex'}}},
                      postaction={decorate}}}
\tikzset{<-/.style = {decoration={markings,
                                  mark=at position 0 with {\arrowreversed[scale=2]{latex'}}},
                      postaction={decorate}}}
\tikzset{<->/.style = {decoration={markings,
                                   mark=at position 0 with {\arrowreversed[scale=2]{latex'}},
                                   mark=at position 1 with {\arrow[scale=2]{latex'}}},
                       postaction={decorate}}}
\tikzset{->-/.style = {decoration={markings,
                                   mark=at position #1 with {\arrow[scale=2]{latex'}}},
                       postaction={decorate}}}
\tikzset{-<-/.style = {decoration={markings,
                                   mark=at position #1 with {\arrowreversed[scale=2]{latex'}}},
                       postaction={decorate}}}
\tikzset{->>/.style = {decoration={markings,
                                  mark=at position 1 with {\arrow[scale=2]{latex'}}},
                      postaction={decorate}}}
\tikzset{<<-/.style = {decoration={markings,
                                  mark=at position 0 with {\arrowreversed[scale=2]{twolatex'}}},
                      postaction={decorate}}}
\tikzset{<<->>/.style = {decoration={markings,
                                   mark=at position 0 with {\arrowreversed[scale=2]{twolatex'}},
                                   mark=at position 1 with {\arrow[scale=2]{twolatex'}}},
                       postaction={decorate}}}
\tikzset{->>-/.style = {decoration={markings,
                                   mark=at position #1 with {\arrow[scale=2]{twolatex'}}},
                       postaction={decorate}}}
\tikzset{-<<-/.style = {decoration={markings,
                                   mark=at position #1 with {\arrowreversed[scale=2]{twolatex'}}},
                       postaction={decorate}}}

\tikzset{circ/.style = {fill, circle, inner sep = 0, minimum size = 3}}
\tikzset{mstate/.style={circle, draw, blue, text=black, minimum width=0.7cm}}

\definecolor{mblue}{rgb}{0.2, 0.3, 0.8}
\definecolor{morange}{rgb}{1, 0.5, 0}
\definecolor{mgreen}{rgb}{0.1, 0.4, 0.2}
\definecolor{mred}{rgb}{0.5, 0, 0}

\def\drawcirculararc(#1,#2)(#3,#4)(#5,#6){%
    \pgfmathsetmacro\cA{(#1*#1+#2*#2-#3*#3-#4*#4)/2}%
    \pgfmathsetmacro\cB{(#1*#1+#2*#2-#5*#5-#6*#6)/2}%
    \pgfmathsetmacro\cy{(\cB*(#1-#3)-\cA*(#1-#5))/%
                        ((#2-#6)*(#1-#3)-(#2-#4)*(#1-#5))}%
    \pgfmathsetmacro\cx{(\cA-\cy*(#2-#4))/(#1-#3)}%
    \pgfmathsetmacro\cr{sqrt((#1-\cx)*(#1-\cx)+(#2-\cy)*(#2-\cy))}%
    \pgfmathsetmacro\cA{atan2(#2-\cy,#1-\cx)}%
    \pgfmathsetmacro\cB{atan2(#6-\cy,#5-\cx)}%
    \pgfmathparse{\cB<\cA}%
    \ifnum\pgfmathresult=1
        \pgfmathsetmacro\cB{\cB+360}%
    \fi
    \draw (#1,#2) arc (\cA:\cB:\cr);%
}
\newcommand\getCoord[3]{\newdimen{#1}\newdimen{#2}\pgfextractx{#1}{\pgfpointanchor{#3}{center}}\pgfextracty{#2}{\pgfpointanchor{#3}{center}}}

\def\Xint#1{\mathchoice
   {\XXint\displaystyle\textstyle{#1}}%
   {\XXint\textstyle\scriptstyle{#1}}%
   {\XXint\scriptstyle\scriptscriptstyle{#1}}%
   {\XXint\scriptscriptstyle\scriptscriptstyle{#1}}%
   \!\int}
\def\XXint#1#2#3{{\setbox0=\hbox{$#1{#2#3}{\int}$}
     \vcenter{\hbox{$#2#3$}}\kern-.5\wd0}}
\def\ddashint{\Xint=}
\def\dashint{\Xint-}

\newcommand\separator{{\centering\rule{2cm}{0.2pt}\vspace{2pt}\par}}

\makeatletter
\@addtoreset{cor-num}{section}
\makeatother

\begin{document}
\maketitle
{\small \centering
\noindent\emph{A brief summary of important ideas and results in the course}

\section{Introduction}
\subsection{Notation and Terminology}
\marginpar{Lecture 1}
\begin{notation}
$X,Y$ sets, then $\P(X)$ is the power set.\\
$[n]=\{1,\cdots,n\}, [m,n]=\{m,\cdots, n\}$\\
Much of the time, $|X|=n$ and $X=[n]$ in this course\\
Level sets: $X^{(r)}=\{A\in\P(X) : |A|=r\}, X^{(<r)}=\{A\in\P(X) : |A|<r\}$\\
$X^{(\omega)}=\{A\in\P(X) : A\ \mathrm{finite}\}$\\
If $G$ is a graph, $E(G)$ is its edge set, $e(G)$ the number of edges, $V(G)$ the vertex set and $|G|$ the number of vertices.  $G$ may also be used to mean its vertex set
\end{notation}


\begin{defi}[Set System]\index{set system}
	$\A\subset \P(X)$ is a \textbf{set system} or \textbf{family of sets} and can be identified with a bipartite graph $G_{\A}(U,W)$ with $U=\A, W=\bigcup_{A\in\A}A$ or $W=X$.
\end{defi}

Note that here the edge $Ax\in E\left(G_{\A}\right)\Leftrightarrow x\in A$. Basically, we're forming a bipartite graph where each vertex in one set is identified with the sets in $\A$, then each vertex in the other are just all the elements that appear in $X$.

\begin{defi}[Set of Distinct Representatives]\index{SDR}\index{set of distinct representatives}
	Given $\A \subset \P(X)$, a \textbf{set of distinct representatives} (an \textbf{SDR}) is an injection $f:\A\rightarrow X$ s.t. $f(A)\in A\ \forall A\in\A$
\end{defi}

If we look at the bipartite graph of an SDR as above, then we see that it corresponds to a complete matching $U\rightarrow W$ (i.e. $\A$ into all elements of $X$ that $\A$ hits)

\begin{thm-num}[Hall, 1935]\index{Hall!theorem}
	A set system $\A$ has an SDR if $\forall\A'\subset\A$:
	$$\left|\bigcup_{A\in\A'}A\right|\geq\left|\A\right|$$
\end{thm-num}

If we look at a bipartite graph in the context of set systems, we see that every bipartite graph corresponds to a multiset of sets, giving us a reformulation of Hall:

\addtocounter{cor-num}{-1}
\begin{thm-num}[Alternate Version]\index{Hall!theorem}
	A bipartite graph $G(U,W)$ has a complete matching if, $\forall S\subset U$,
	\begin{equation}\index{Hall!criterion}\label{eq:hc}
		\left|\Gamma(S)\right|\geq|S|
	\end{equation}
	Where $\Gamma(S)$ is the set of neighbours of $S$.
\end{thm-num}
\begin{proof}[Proof 1]
	The necessary direction is trivial, so we look only at sufficiency.
	
	Assume $G$ is edge minimal w.r.t. (\ref{eq:hc}). Here we look for a contradiction - suppose that $G$ is not a matching, so we can find two edges out of one vertex in $U$ - say $uw_1$ and $uw_2$. Use minimality to get two sets $S_1, S_2$ in $U$ with the same number neighbours as vertices and $u$ is the only neighbour of $w_i$ in $S_i$.
	
	Then, use $\left|S_1\cap S_2\right|\leq\left|\Gamma\left(S_1\cap S_2\right)\right|$ and manipulate the RHS using conditions on the $S_i$ and their intersection and minimality to get $\left|\Gamma\left(S_1\cap S_2\right)\right|\leq\left|S_1\cap S_2\right|$, so those two expressions are equal.
	
	Then, delete $u$ from $S_1\cap S_2$ and look at that set's size (and the size of its neighbours) to get a contradiction.
\end{proof}
\begin{proof}[Proof 2]
	The necessary direction is trivial, so we look only at sufficiency.
		
	Assume $G$ is edge minimal w.r.t. (\ref{eq:hc}). Here we induct. Look at $\xi=\{E\subset U : \left|\Gamma(E)\right|=|E|>0\}\neq\phi$.
	
	Split into the cases about whether or $\exists E\in \xi$ where $E\neq U$. If it exists, then form the sub-graphs from $H=E\cup\Gamma(E)$ and $G\smallsetminus H$ - both satisfy (\ref{eq:hc}) so done by induction. Otherwise, pick an edge $uw$ - then $G\smallsetminus \{u,w\}$ satisfies (\ref{eq:hc}) so add back $u$ and $w$ and we're done.
\end{proof}

\begin{cor-num}
	$G(U,W)$ bipartite, $d(u)\geq d(w)\ \forall u\in U, w\in W$. Then $\exists$ a complete matching $U\rightarrow W$.
\end{cor-num}
\begin{proof}
	Pick $d(u)\geq d\geq d(w)$, look at a general $S\subset U$ to get $d|S| \leq e(S, \Gamma(S)) \leq d|\Gamma(S)|$
\end{proof}


\begin{defi}[$(r,s)$-regular]\index{regular}
	\marginpar{Lecture 2}
	A bipartite graph $G(U, W)$ is \textbf{$(r,s)$-regular} if $d(u)=r$ and $d(w)=s\quad\forall u\in U,\ \forall w\in W$
\end{defi}

Note that $r|U|=e(G)=s|W|$ and that Corollary 2 implies that $G(U, W)$ being $(r,s)$-regular gives us a complete matching from $U$ to $W$ if $|U|\leq |W|$

\begin{cor-num}\index{matching!complete}
	Let $0\leq i,j\leq n$, with $\binom{n}{i}\leq\binom{n}{j}$. Then $\exists$ a complete matching $f:[n]^{(i)}\rightarrow[n]^{(j)}$ s.t. $f(A)\subset A$ if $j\leq i$ and $f(A)\supset A$ if $i\leq j$.
\end{cor-num}

\begin{thm-num}
	Let $G=G(U,W)$ be a connected $(r,s)$-regular graph. Then, for $\phi\neq A\subset U$,
	
	$$\frac{|\Gamma(A)|}{|W|}\geq \frac{|A|}{|U|}$$
	
	With equality iff $A=U$
\end{thm-num}
\begin{proof}
	Use $r|A|=e(A,\Gamma(A))\leq s|\Gamma(A)|$ and divide by $|W|$ to get the result (noting $r|U|=s|W|$). If equality holds, then all edges out of $\Gamma(A)$ go into $A$, so $A=U$ or the else graph is disconnected.
\end{proof}

\begin{defi}[Partially Ordered Set]\index{poset}
	A \textbf{partially ordered set}, or \textbf{poset} is a set with a binary relation $\leq$ that satisfies reflexivity ($a\leq a$), antisymmetry ($a\leq b, b\leq a \Rightarrow a=b$) and transitivity ($a\leq b, b\leq c \Rightarrow a\leq c$)
\end{defi}

\begin{defi}[The Cube]\index{cube}
	We define \textbf{the cube}: $Q^n\cong\P(n)\cong[2]^n\cong$ the set of all $0-1$ sequences.
\end{defi}
\begin{rem}
	\begin{itemize}
		\item $Q^n$ can be considered as a graph - $AB$ is an edge iff $|A\Delta B|=1$ (look at an $n$ dimensional cube with vertices containing all coordinates with 0 or 1 and put $j\in A$ if $x_j=1$)\\
		\item It is also a poset via $A<B$ if $A\subset B$
		\item $Q^n$ has a natural orientation: $\overrightarrow{AB}$ if $A=B\cup\{a\}$
		\item For the order induced on $Q^n$, see Dilworth's Theorem
	\end{itemize}
\end{rem}

\section{Sperner Systems}
\begin{defi}[Sperner]\index{Sperner}
	A set system $\A\subset\P(n)$ is \textbf{Sperner} if $A,B\in\A,\ A\neq B\Rightarrow A\not\subset B$. That is no two non-equal sets in $\A$ contain each other.
\end{defi}

The simplest example of Sperner sets are the level sets, $X^{(r)}$

\begin{defi}[Chain]\index{chain}
	In a poset, a \textbf{chain} is a linearly ordered set, $c_1<c_2<\cdots<c_m$
\end{defi}

\begin{defi}[Antichain]\index{chain!antichain}
	In a poset, an \textbf{antichain} has no two elements that are comparable.
\end{defi}

\begin{thm-num}[Sperner, 1928]\index{Sperner!Theorem}
	If $\A\subset\P(n)$ is Sperner, then $|\A|\leq\binom{n}{\lf n/2\rf}$
\end{thm-num}
\begin{proof}
	For $0\leq r\leq n-1$, we can easily find a complete matching between $X^{(r)}$ and $X^{(r+1)}$ from the smaller set to the larger set (where $AB\in E\left(X^{(r)},X^{(r+1)}\right)$ iff $A\subset B$ or $B\subset A$). All these matchings put together create a collection of $\binom{n}{\lf n/2\rf}$ paths in $Q^n$ (look from the largest layer in the middle outward - i.e. the layer where $r=\lf n/2\rf$), each of which is a chain. Hence our poset is covered by $\binom{n}{\lf n/2\rf}$, giving $|\A|\leq\binom{n}{\lf n/2\rf}$
\end{proof}}

\begin{defi}[Weight]\index{weight}
	The \textbf{weight} $w(A)$ of a set $A\in\P(n)$ is $w(A)=\frac{1}{\binom{n}{|A|}}$
\end{defi}

\begin{thm-num}[LYM or YMBL\footnote{Yamamoto, 1954; Meshalkin, 1963; Bollob\'as, 1965; Lubell, 1966}] \index{Yamamoto}\index{Meshalkin}\index{Bollob\'as}\index{Lubell}\index{LYM}\index{YMBL}
	Let $\A$ be a Sperner system on $X$, with $|X|=n$. Then $$w(\A):=\sum_{A\in\A}w(A)\leq 1$$
\end{thm-num}
\begin{proof}
	Look at the maximal chains in $\P(X)$, $A_0\subset A_1\subset\cdots\subset A_n$, with $A_i\in X^{(i)}$ (so $|A_i|=i$). By just picking elements one at a time, we can see that the number of maximal chains is $n!$. Each such chain has $\leq 1$ element of $\A$, as $\A$ is Sperner. But also, every $A\in\A$ is in $|A|!\left(n-|A|\right)!$ chains.
	
	Hence, $\sum_{A\in\A}|A|!\left(n-|A|\right)!\leq n!$ which gives the result.
\end{proof}

\addtocounter{cor-num}{-1}
\begin{thm-num}[Alternate Version]\index{LYM}\index{YMBL}\marginpar{Lecture 3}
	If $\A\subset \P(n)$ is an anti-chain, then $w(\A)\leq 1$
\end{thm-num}

\begin{cor-num}
	If $\A\subset\P(n)$ is a Sperner system, then $|\A|\leq\binom{n}{\lf n/2\rf}$, with equality iff $\A$ is $X^{(\lf n/2\rf)}$ or $X^{(\lc n/2\rc)}$.
\end{cor-num}
\begin{proof}
	The inequality is immediate.
	
	With equality, $\A\subset X^{(\lf n/2\rf)}\cup X^{(\lc n/2\rc)}$ so the claim holds for $n$ even - now we check $n=2k+1$. Write $\A_k=\A\cup X^{(k)}, \A_{k+1}=\A\cup X^{(k+1)}$, which yields a bipartite graph (if both non-empty, else claim holds).
	
	Then, using Theorem 4 from Section 1, $|\A|=\left|\A_k\right|+\left|\A_{k+1}\right|<\left|\A_k\right|+\left|\Gamma\left(\A_{k+1}\right)\right|\leq\binom{n}{k}$.
\end{proof}

\begin{defi}[$k$-Sperner]\index{Sperner!$k$-Sperner}
	We say that $\A\subset\P(n)$ is $k$-Sperner if it does not contain $A_1\subsetneqq A_2\subsetneqq \cdots\subsetneqq A_{k+1}$ - i.e. it doesn't contain a chain of length $k+1$.
\end{defi}

Note that a Sperner system is 1-Sperner.

\begin{cor-num}[Erd\H{o}s, 1945]\index{Erd\H{o}s}
	If $\A\in\P(n)$ is $k$-Sperner, then $|\A|$ is at most the sum of the $k$ largest binomial coefficients.
\end{cor-num}

\begin{thm}[Littlewood and Offord - LO\footnote{I don't think this is examinable}]\index{Littlewood}\index{Offord}
	$z_i\in\C$, $\left|z_i\right|\geq 1$, then the number of $\sum_1^n\pm z_i$ within $k$ of each other is $\leq ck\frac{2^n\log n}{\sqrt{n}}$ for some constant $c$.
\end{thm}

\begin{thm-num}[Erd\H{o}s, 1945]\index{Erd\H{o}s}
	Let $x_1,\cdots, x_n\in\R$ with $x_i\geq1$. Then the number of sums $\sum_1^n\pm x_i$ in an open interval $J$ of length $2k$ is at most the sum of the $k$ largest binomial coefficients.
\end{thm-num}
\begin{proof}
	Define $\A=\{A\in\P(n) : \sum_{i\in A}x_i - \sum_{j\notin A}x_j\in J\}$ - this is $k$-Sperner, as moving $x_i$ from one sum to the other gives a change in value of at least 2, so a chain of $k+1$ sets would need a change of at least $2k$. Hence, we're done.
\end{proof}

\begin{conj}[Erd\H{o}s, 1945]\index{Erd\H{o}s}
	If $x_1,\cdots,x_n\in E$, where $E$ is a normed space and $\left|\left|x_i\right|\right|\geq1$, then the number of sums $\sum x_i$ less than distance 2 from each other is $\binom{n}{\lf n/2 \rf}$
\end{conj}

\begin{defi}[Symmetric Chain]\index{chain!symmetric}
	A chain $A_0\subset A_1\subset \cdots \subset A_k$ is \textbf{symmetric} if $\left|A_{i+1}\right|=\left|A_{i}\right|+1\ \forall i$ and $\left|A_0\right|+\left|A_k\right|=n$
\end{defi}

\begin{thm-num}[Kleitman and Katona]\index{Kleitman}\index{Katona}
	$\P(n)$ has a decomposition into symmetric chains.
\end{thm-num}
\begin{proof}
	Induct on $n$ - the base case is $\{\phi,\{1\}\}$.
	
	If $\P(n-1)=\dot{\bigcup}_{i=1}^{q}\calC_i$ with $\calC_i = \left\{A_0\subset\cdots A_k\right\}$, then form $$\calC_i'=\left\{A_0,\cdots,A_{k-1},A_k\cup\{n\}\right\}$$
	and $$\calC_i''=\left\{A_0\cup\{n\},\cdots,A_{k-1}\cup\{n\}\right\}$$
	Then $\calC_i'$ and $\calC_i''$ partition $\P(n)$ into symmetric chains.
\end{proof}
\begin{rem}\marginpar{Lecture 4}
	Note that we've formed a partition into $\binom{n}{\lf n/2\rf}$ symmetric chains, with $\binom{n}{i}-\binom{n}{i-1}$ chains of length $n+1-2i$ (so 1 chain of length $n+1$, $n-1$ chains of length $n-1$, etc.)
\end{rem}

For the next few results, let $E$ be a normed space, $x_1,\cdots, x_n\in E$, $\left|\left|x_i\right|\right|\geq 1\forall i$ and, for $A\in\P(n)$, $x_A=\sum_{i\in A}x_i$

\begin{defi}[Scattered]\index{set system!scattered}
	Call $\calD\subset\P(n)$ \textbf{scattered} if $\left|\left|x_A - x_B\right|\right|\geq1\ \forall A, B\in \calD$.
\end{defi}

\begin{defi}[Symmetric Partition]\index{symmetric}
	Call a partition $\P(n)=\dot{\bigcup}_1^s\calD_j$ \textbf{symmetric} if there are precisely $\binom{n}{i}-\binom{n}{i-1}$ sets $\calD_i$ of cardinality $n+1-2i$.
\end{defi}

\begin{thm-num}[Kleitman, 1970]\index{Kleitman}
	$E$ and $(x_i)_1^n$ as above. Then, $\P(n)$ has a symmetric partition into scattered sets.
\end{thm-num}
\begin{proof}\footnote{Different in lectures - this avoids support functionals}
	Induct on $n$ - the base case is $\{\phi,\{1\}\}$.
	
	Otherwise, let $\P(n-1)=\dot{\bigcup}_{i=1}^{q}\calD_i$ be a scattered partition into scattered sets and let $\calD_j=\{A_1,\cdots,A_m\}$.
	
	In each $\calD_j$, pick $A_m$ so that $\left|\left|x_{A_m} + x_n - x_{A_j}\right|\right|\geq \left|\left|x_n\right|\right|\ \forall\ 1\leq j<m$. This is possible - indeed, if not, then we can find a function $g:[n]\rightarrow[n]$ such that $\left|\left|x_{A_i} + x_n - x_{A_{g(i)}}\right|\right|< \left|\left|x_n\right|\right|\ \forall i\in[n]$ and $g(i)\neq i$. It's clear then that we must be able to find $k>1$ and $r\in[n]$ with $g^k(r)=r$. But then $\left|\left|x_{A_{g^i(r)}} + x_n - x_{A_{g^{i+1}(r)}}\right|\right|< \left|\left|x_n\right|\right|\ \forall\ 0\leq i<k$. Add all these up and divide the LHS by $k$ to get a contradiction via the triangle inequality.
	
	Then, write
	\begin{align*}
	\calD_j'&=\{A_1,\cdots,A_m,A_m\cup\{n\}\}\\
	\calD_j''&=\{A_1\cup\{n\},\cdots,A_{m-1}\cup\{n\}\}
	\end{align*}
	
	That $\calD_j'$ and $\calD_j''$ form a symmetric partition is clear. To see scattered, note that it's trivially true for $\calD_j''$ and most of $\calD_j'$ - we only need to check the case $\left|\left|x_{A_m\cup\{n\}}-x_{A_j}\right|\right|\geq 1$. But this is true by our choice of $A_m$ above - indeed $\left|\left|x_{A_m\cup\{n\}} - x_{A_j}\right|\right|=\left|\left|x_{A_m} + x_n - x_{A_j}\right|\right|\geq \left|\left|x_n\right|\right|\geq1$
\end{proof}

\begin{thm-num}[Kleitman\footnote{A conjecture of Erdo\H{o}s in 1945}, 1970]\index{Kleitman}
	If $\A\subset\P(n)$ s.t. $\forall A,B\in\A$, $\left|\left|x_A - x_B\right|\right|<1$, then $|\A|\leq\binom{n}{\lf n/2\rf}$
\end{thm-num}
\begin{proof}
	Take a symmetric partition $\P(n)=\dot{\bigcup}_{j=1}^{s}\calD_j,\ s=\binom{n}{\lf n/2\rf}$ into scattered sets.
	
	Then $\left|\A\cup\calD_j\right|\leq 1$ because of the scattered condition, so $|\A|\leq s=\binom{n}{\lf n/2\rf}$.
\end{proof}

\section{The Kruskal-Katona Theorem}
\begin{defi}[Lower Shadow]\index{set system!lower shadow}
	If $\A\subset X^{(r)}$, then the \textbf{lower shadow of $\A$}, is given by $\partial\A=\{B\in X^{(r-1)}:B\subset A \textrm{ for some } A\subset\A \}$	
\end{defi}

\begin{fact}
	$|\partial\A|\geq|\A|\binom{n}{r-1}/\binom{n}{r}=|\A|\frac{r}{n-r+1}$ with equality iff $\A$ is $\phi$ or $X^{(r)}$.
\end{fact}

We also ask the in-between question - what is $\calB\subset X^{(r)}$ s.t. $|\calB|=|\A|$ and $|\partial\calB|\leq|\partial\A|$?

It's clear that $\partial[k]^{(r)}=[k]^{(r-1)}$, so $\exists\calB_1,\calB_2,\cdots\subset X^{(r)}$ s.t. $|\calB_m|=m$ and $\left|\partial\calB_m\right|\leq\left|\partial\calA\right|\ \forall\calA\subset X^{(r)}, |\calA|=m$

\begin{defi}[Lexicographical Order]\index{ordering!lexicographical}
	In the lexicographical order, $A<B$ if $\min(A\Delta B)\in A$. It can be summarised as ``prioritise using small numbers".
\end{defi}
\begin{defi}[Colex Order]\index{ordering!colex}
	In the colex order, $A<B$ if $\max(A\Delta B)\in B$, or alternatively if $\sum_{i\in A}2^i<\sum_{j\in B}2^j$. It can be summarised as ``prioritise using small numbers".
\end{defi}
\begin{eg}[Lex and Colex orders]
	For $n=6,r=3$ we get the total orders:
	
	Lex: $123,124,125,126,134,135,136,145,146,156,234,\cdots, 456$
	
	Colex: $123,124,134,234,125,135,235,\cdots, 456$
\end{eg}

\printindex
\end{document}
