\documentclass[a4paper]{article}

\def\npart {III}
\def\nterm {Michaelmas}
\def\nyear {2016}
\def\nlecturer {C. Johansson}
\def\ncourse {Local Fields}

\input{../summary_header}

\begin{document}
\maketitle
{\small \centering
\noindent\emph{A brief summary of important ideas and results in the course}

\setcounter{section}{-1}
\section{Introduction}
\marginpar{Lecture 1}
If we look at $f(x_1, \cdots, x_n)\in\Z[x_1,\cdots,x_n]$, what are the ways we can look for solutions $\mathbf{a}\in\Z^n$?

One way would be to look over $\R$, but the point of this course is to package all of the information modulo $p^n\forall n\geq0$ together. 

\begin{notation}
	Throughout this course, all rings will be commutative with a 1, unless otherwise stated.
\end{notation}

\section{Basic Theory}

\subsection{Some Generalities}
\begin{defi-num}[Absolute value]\index{absolute value}
	Let $K$ be a field. An \textbf{absolute value} on $K$ is a function $|\cdot|:K\rightarrow\R_{\geq 0}$ s.t.
	\begin{enumerate}
		\item $|x|=0\iff x=0$\\
		\item $|xy|=|x|\cdot|y|\quad\forall x,y\in K$\\
		\item $|x+y|\leq|x|+|y|\quad\forall x,y\in K$
	\end{enumerate}
\end{defi-num}
\begin{eg}
	$\Q,\R,\C$ with $|z|=\sqrt{z\overline{z}}$
\end{eg}
Note that $\left||x| - |y|\right|\leq|x-y|\ \forall x,y$. Also, an absolute value defines a metric $d(x,y)=|x-y|$\index{absolute value!metric} on $K$

\begin{defi-num}[Valued Field]\index{valued field}
	A \textbf{valued field} is a field with an absolute value.
\end{defi-num}

\begin{defi-num}[Equivalent]\index{valued field!equivalent}\label{def:equivalent}
	If $K$ is a field, then two absolute values $|\cdot|,|\cdot|'$ are \textbf{equivalent} if they induce the same topology.
\end{defi-num}

\begin{exer-num}
	Using notation as in Definition \ref{def:equivalent}, prove that TFAE
	
	\begin{enumerate}
		\item $|\cdot|$ and $|\cdot|'$ are equivalent\\
		\item $\forall x\in\quad|x|<1\Rightarrow|x|'<1$\\
		\item $\exists s\in\R_{>0}$ s.t. $|x|^s=|x|'\quad x\in K$
	\end{enumerate}
\end{exer-num}

\begin{exer-num}
	Let $K$ be a valued field. Then the completion $\hat{K}$ of K is independent of $|\cdot|$ up to equivalence, and it is a valued field with an absolute value extending $|\cdot|$.
\end{exer-num}

\begin{defi-num}[Archimedean]\index{Archimedean}\index{Archimedean!non}\index{triangle inequality!strong}
	An absolute value $|\cdot|$ on a field $K$ is called \textbf{non-Archimedean} if it satisfies the strong triangle inequality, i.e.
	\[
		|x+y|\leq \max(|x|,|y|)
	\]
	
	Otherwise, the absolute value is \textbf{Archimedean}.
\end{defi-num}

Unless otherwise mentioned, \textbf{all absolute values will be non-Archimedean}. Also, all \textbf{absolute values are assumed to be non-trivial}.\index{absolute value!assumptions}

\begin{defi}\index{valuation ring}
	If $K$ is a valued field, then the \textbf{valuation ring} of $K$ is $\calO=\{x\ :\ |x|\leq 1\}$.
\end{defi}

\begin{prop-num}
	\begin{enumerate}
		\item $\calO$ is an open subring of $K$\\
		\item $\forall r\in(0,1]$, $\{x\ :\ |x|<r\}$ and $\{x\ :\ |x|\leq r\}$ are open ideals of $\calO$\\
		\item $\calO^{\mathrm{x}}=\{x\ :\ |x|=1\}$
	\end{enumerate}
\end{prop-num}
\begin{proof}
	Fairly trivial - obvious proof for each section works.
\end{proof}

\begin{prop-num}
	Let $K$ be a valued field. For parts (ii) and (iii), assume that $K$ is complete.
	\begin{enumerate}
		\item Let $(x_n)$ be a sequence in $K$. If $x_n-x_{n+1}\rightarrow0$, then $(x_n)$ is Cauchy.\\
		\item Let $(x_n)$ be a sequence in $K$. If $x_n-x_{n+1}\rightarrow0$, then $(x_n)$ converges.\\
		\item Let $\sum_{n=0}^{\infty}y_n$ be a series in $K$. If $y_n\rightarrow0$, then $\sum_{n=0}^{\infty}y_n$ converges.
	\end{enumerate}
\end{prop-num}
\begin{proof}
	The first follows from the Archimedean assumption - use epsilons and that: $$\left|x_m-x_n\right|=\left|x_m-x_{m-1}+x_{m-1}-\cdots-x_n\right|\leq\max(\left|x_m-x_{m-1}\right|,\cdots,\left|x_{n+1}-x_{n}\right|)$$
	The other two follow easily from the first.
\end{proof}

\begin{defi-num}[Integral Over a Ring]\index{ring!integral}\marginpar{Lecture 2}
	Let $R\subseteq S$ be rings, then $s\in S$ is \textbf{integral over $R$} if there exists a monic $f(x)\in R[x]$ s.t. $f(s)=0$.
\end{defi-num}

\begin{prop-num}
	Let $R\subseteq S$ be rings. Then, $s_1,\cdots,s_n\in S$ are integral over $R\iff R[s_1,\cdots,s_n]\subseteq S$ is a finitely generated $R$-module.
\end{prop-num}
\begin{proof}
	We do the $\Rightarrow$ direction first.
	
	By induction, it suffices to prove the case $n=1$. Pick a monic poly with $f(s)=0$, and construct any other polynomial using this and the division algorithm, so that $1,s,\cdots,s^{\deg f-1}$ is a basis of $R[s]$.
	
	For the $\Leftarrow$ direction, pick $R$-module generators and an element from $b\in R[s_1,\cdots,s_n]$. Write $bt_i$ in terms of the $t_j$ to make a matrix, then use its determinant and an "inverse" (adjoint) to get a polynomial from the determinant.
\end{proof}

\begin{cor-num}
	Let $R\subseteq S$ be rings. If $s_1, s_2$ are integral over $R$, then $s_1+s_2$ and $s_1s_2$ are integral over $R$.
	
	Moreover, the set $\widetilde{R}\subseteq S$ of all elements of $S$ integral over $R$ is a ring.
\end{cor-num}
\begin{defi}[Integral Closure]\index{integral clsoure}\index{integrally closed}
	$\widetilde{R}$ is called the \textbf{integral closure} of $R$ in $S$. If $R=\widetilde{R}$, then we say $R$ is integrally closed in $S$.
\end{defi}
\begin{proof}
	$s_1s_2$ integral over $R$, so $R[s_1,s_2]$ finite over $R$ and hence any $b\in R[s_1, s_2]$ is integral over $R$.
\end{proof}

\begin{defi-num}[Ring Topology]\index{ring!topology}\index{ring!topological}
	Let $R$ be a ring. A topology on $R$ is called a \textbf{ring topology} on $R$ if addition and multiplication are a continuous map $R\times R\to R$, where $R\times R$ is given the product topology.
	
	A ring with a ring topology is called a \textbf{topological ring}.
\end{defi-num}

\begin{exer}
	Let $K$ be a valued field. Then $K$ is a topological ring.
\end{exer}

\begin{defi-num}[$I$-adically Open]\index{I-adic!open}
	Let $R$ be a ring, $I\subseteq R$ an ideal. A subset $U\subseteq R$ is called \textbf{$I$-adically open} if $\forall x\in U,\ \exists n\geq 1$ s.t. $x+I^n \subseteq U$.
\end{defi-num}

\begin{prop-num}
	The set of all $I$-adically open sets form a topology on $R$, call the $I$-adic topology\index{I-adic!topology}.
\end{prop-num}
\begin{proof}
	$\phi, R$ are $I$-adically open by definition, as are arbitrary unions of $I$-adically open sets. Also, if $U,V$ are $I$-adically open and $x\in U\cap V$ with $x+I^m\subseteq U$ and $x+I^n\subseteq V$, then $x+I^{\max(m,n)}\subseteq U\cap V$
\end{proof}

\begin{exer}
	The $I$-adic topology is a ring topology of $R$.
\end{exer}

\begin{defi-num}[Inverse Limit]\index{inverse limit}\index{projective limit}
	Let $R_1, R_2,\cdots$ be topological rings with continuous homomorphisms $f_n : R_{n+1}\to R_n\ \forall n\geq 1$. Then, the \textbf{inverse limit} or \textbf{projective limit} of the $R_i$ is the ring:
	\[
		\xleftarrow[\mathrm{n}]{\mathrm{lim}} = \left\{\left(x_n\right)\in\prod_nR_n\ :\ f_n\left(x_{n+1}\right)=x_n\ \forall n\geq 1\right\}
		\subseteq \prod_nR_n
	\]
	with coordinate-wise addition/multiplication, together with the subspace topology\footnote{called the inverse limit topology} induced by the product topology on $\prod_nR_n$.
\end{defi-num}

\begin{prop-num}
	The inverse limit topology is a ring topology.
\end{prop-num}
\begin{proof}
	Use that $\prod R_n\times\prod R_n\to\prod R_n$ is continuous, with the map being coordinate-wise addition/multiplication, as well as containment of the inverse limit inside $\prod R_n$ and the projection map. Also, use the map $\left(\xleftarrow[\mathrm{n}]{\mathrm{lim}}R_n\right)\times\left(\xleftarrow[\mathrm{n}]{\mathrm{lim}}R_n\right)\to\prod R_n\times\prod R_n$
\end{proof}

\begin{defi-num}[$I$-adic completion]\index{I-adic!completion}
	$R$ a ring, $I$ an ideal. The \textbf{$I$-adic completion} of $R$ is the topological ring:
	\[
		\xleftarrow[\mathrm{n}]{\mathrm{lim}} R/I^n	
	\]
	Where $R/I^n$ has the discrete topology and $R/I^{n+1}\to R/I^n$ is the natural map.
\end{defi-num}

$\calA \subset X^{\lf{n/2}\rf}\cup X^{\lc{n/2}\rc}$

\begin{defi}[$I$-adically Complete]\index{I-adic!complete}\marginpar{Lecture 3}
	$R$ a ring, $I\subseteq R$ an ideal. Then $\exists$ a map 
	\begin{align*}
		\nu : R &\to \left(\xleftarrow[\mathrm{n}]{\mathrm{lim}}R/I^n\right)\\
		r & \mapsto \left(r\ \mathrm{mod}\ I^n\right)_n
	\end{align*}
	This map is a cts ring homomorphism when $R$ is given the $I$-adic topology. We say that $R$ is \textbf{$I$-adically complete} if $\nu$ is a bijection.
\end{defi}

\begin{exer}
	If $\nu$ is a bijection, then $\nu$ is a homeomorphism
\end{exer}

If $I=xR$, then we often call the $I$-adic topology the $x$-adic topology.

\section{The $p$-adic numbers}
From now on in this course, $p$ will be taken to mean a prime number.

If $x\in\Q \setminus \{0\}$, then there is a unique representation $p^n\frac{a}{b}$, where $a,n\in\Z,\ b\in\Z_{\geq 0}$ and $(a,p)=(b,p)=(a,b)=1$

\begin{defi}[$p$-adic Absolute Value]\index{p-adic!absolute value}
	We define the \textbf{$p$-adic absolute value} on $\Q$ to be the function $|\cdot|\ :\ \Q\to\R_{\geq0}$ given by:
	\[
		|x|_p =
		\begin{cases}
			0, & \text{if } x=0\\
			p^{-n}, & \text{if } x=p^n\frac{a}{b}\quad(\text{i.e. }x\neq0)
		\end{cases}
	\]
\end{defi}

That $|\cdot|$ is a (non-Archimedean) absolute value is clear.

\begin{fact}
	$x\in\Z_{\neq0}\implies|x|_p=p^{-n}\iff p^n\|x$
\end{fact}

\begin{defi-num}[$p$-adic Numbers/Integers]\index{$p$-adic!numbers}\index{$p$-adic!integers}
	The \textbf{$p$-adic numbers}, $\Q_p$ is the completion of $\Q$ with respect to $|\cdot|_p$
	
	The valuation ring $\Z_p=\{x\in\Q_p\ :\ |x|_p\leq1\}$ is called the \textbf{$p$-adic integers}.
\end{defi-num}

\begin{prop-num}
	$\Z_p$ is the closure of $\Z$ inside $\Q_p$.
\end{prop-num}
\begin{proof}
	Consider that $\Z_{(p)}=\{x\in\Q\ :\ |x|_p\leq1\}$ is dense in $\Z_{(p)}$ and that $\Z\subseteq\Z_{(p)}$. Show that $\Z$ is dense in $\Z_{(p)}$.
\end{proof}

\begin{prop-num}
	The non-zero ideals of $\Z_p$ are $p^n\Z_p$ for $p\geq0$. Moreover,
	\[
		\Z/p^n\Z\cong\Z_p/p^n\Z_p
	\]
\end{prop-num}
\begin{proof}
	Pick an ideal $I$ and a maximal element $x\neq0$ from it. Show that $I=xR$ and $p^n\Z_p=x\Z_p$ to get the first part.
	
	For the second part, look at $f_n\ :\ \Z\to\Z_p/p^n\Z_p$ and its kernel (which is $p^n\Z$), showing that it induces an isomorphism.
\end{proof}

\begin{cor-num}
	$\Z_p$ is a PID with a unique prime element $p$ (up to units).
\end{cor-num}

\begin{prop-num}\index{$p$-adic!topology}
	The topology on $\Z$ induced by $|\cdot|_p$ is the $p$-adic topology
\end{prop-num}
\begin{proof}
	Straightforward - pick $U\subseteq\Z$ and show it directly.
\end{proof}

\begin{prop-num}
	$\Z_p$ is $p$-adically complete and is isomorphic to the $p$-adic completion of $\Z$.
\end{prop-num}
\begin{proof}
	The second part follows from the first by the proof of Proposition 20 via
	\[
		\Z_p \xleftarrow{\nu} \left(\xleftarrow[\mathrm{n}]{\mathrm{lim}}\Z_p/p^n\Z_p\right) 
		     \xrightarrow{\nu} \lim\Z/p^n\Z
	\]
	
	To prove the first part, we get injectivity by looking at $x\in\ker(\nu)\iff x\in p^n\Z_p\ \forall n\iff x=0$.
	
	We get surjectivity by picking $(z_n)\in\xleftarrow[\mathrm{n}]{\mathrm{lim}}\Z_p/p^n\Z_p$ with the unique representative of $z_n$ in $\{0,1,\cdots,p^n-1\}$ as $x_n=\sum_{i=0}^{n-1}a_ip_i$, then considering $x=\sum_{i=0}^{\infty}a_ip^i$.
\end{proof}

\begin{cor-num}
	\marginpar{Lecture 4}
	Every $a\in\Z_p$ has a unique expansion $a=\sum_{i=0}^{\infty}a_ip^i$ with $a_i\in\{0,1,\cdots,p-1\}$
	
	Also, every $a\in\Q_p^{\mathrm{x}}$ has a unique expansion $a=\sum_{i=n}^{\infty}a_ip^i$ with $n\in\Z,\quad n=-\log_p|a|_p$ and $a_n\neq0$.
\end{cor-num}


\section{Valued Fields}

\begin{defi-num}[Valuation]\index{valuation}
	Let $K$ be a valued field. A \textbf{valuation} on $K$ is a function $V\ :\ K\to\R\cup\{\infty\}$ s.t., $\forall x,y\in K$:
	\begin{enumerate}
		\item $v(x)=\infty \iff x=0$\\
		\item $v(xy)=v(x)+v(y)$\\
		\item $v(x+y)\geq\min(v(x),v(y))$
	\end{enumerate}
	Here we use the conventions $r+\infty=\infty,\ r\leq\infty\quad\forall r\in \R\cup\{\infty\}$
\end{defi-num}

\begin{rem}
	If $V$ is a valuation, and $|x|=c^{-V(x)}$ for some $c\in\R_{>1}$, then $|\cdot|$ is an absolute value.
	
	Conversely, $|\cdot|$ an absolute value implies that $V(x)=-\log_c|x|$.
\end{rem}

\begin{defi}[Field of Formal Laurent Series']\index{field of formal Laurent series}
	If $K$ a field, then the \textbf{field of formal Laurent series' over $K$} is 
	\[
		K((T))=\left\{\sum_{i\gg-\infty}^{\infty}a_iT^i\ :\ a_i\in K\right\}
	\]
\end{defi}

\begin{exer}
	Show that $v\left(\sum a_iT^i\right)=\min\{i\ :\ a_i\neq0\}$ is a valuation of $K((T))$
\end{exer}

\begin{notation}
	The \textbf{valuation ring}: $\calO=\calO_k=\{x\in K\ :\ |x|\leq 1\}$\index{p-adic!valuation ring}\\
	The \textbf{maximal ideal}: $\mathfrak{m}=\mathfrak{m}_k=\{x\in K\ :\ |x|< 1\}$\index{p-adic!maximal ideal}\\
	The \textbf{residue field}: $k=k_K=\calO/\mathfrak{m}$\index{p-adic!residue field}
\end{notation}

\begin{defi}[Primitive]\index{primitive}
	If $K$ a valued field and $a_0+a_1x+\cdots a_nx^n=F(x)\in K[x]$ is a polynomial, we say $F$ is \textbf{primitive} if $\max_i|a_i|=1$ (so $F\in\calO[x]$).
\end{defi}

\begin{thm-num}[Hensel's Lemma]\index{Hensel's lemma}
	Assume that $K$ is complete and that $F\in K[x]$ is primitive. Put $f= F\ \mathrm{mod}\ \mathfrak{m}\in\mathfrak{k}[x]$. Then, if there's a factorisation $f(x)=g(x)h(x)$ with $(g,h)=1$, then there's a factorisation $F(x)=G(x)H(x)$ in $\calO[x]$ with $g\equiv G,\ h\equiv H\ \mathrm{mod}\ \mathfrak{m}$ and $\deg g = \deg G$.
\end{thm-num}
\begin{proof}
	Put $d=\deg F, m=\deg g$, giving $\deg h\leq d-m$. Pick lifts $G_0,H_0\in\calO[x]$ of $g,h$ with $\deg G_0=\deg g$ and $\deg H_0\leq d-m$.
	
	So, $(g,h)=1\implies\exists A,B\in\calO[x]$ s.t. $AG_0+BH_0\equiv1\mod{\mathfrak{m}}$
	
	Pick $\pi\in\mathfrak{m}$ s.t. $F=G_0H_0\equiv AG_0+BH_0-1\equiv0\mod{\pi}$. Then, we want to find:
	
	\[
		\begin{rcases}
			G &= G_0+\pi P_1+\pi^2 P_2+\cdots\\
			H &= H_0+\pi Q_1+\pi^2 Q_2+\cdots
		\end{rcases} \in\calO[x]
	\]
	with $P_i,Q_i\in\calO[x]$, $\deg P_i<m$ and $\deg Q_i\leq d-m$. By the definition of $\mathfrak{m}$, this would converge, so doing this suffices.
	
	We want $F\equiv G_{n-1}H_{n-1}\mod{\pi^n}$ then to take the limit, which we'll find by induction. The base case is trivial, so we assume we have $G_{n-1},H_{n-1}$. We want to find $G_n=G_{n-1}+\pi^n P_n$ and $H_n=H_{n-1}+\pi^n Q_n$.
	
	Expanding $F-G_nH_n$, it is equivalent to find
	\[
		F-G_{n-1}H_{n-1}\equiv\pi^n\left(G_{n-1}Q_n+H_{n-1}P_n\right)\mod{\pi^{n+1}}
	\]
	
	Divide by $\pi^n$ to get:
	\[
		G_0Q_n+H_0P_n\equiv G_{n-1}Q_n+H_{n-1}P_n\equiv\frac{1}{\pi^n}\left(F-G_{n-1}H_{n-1}\right)\mod{\pi}
	\]
	
	But $AG_0+BH_0\equiv1\mod{\pi}\implies F _n\equiv AG_0F_n+BH_0F_n\mod{\pi}$. Now, write $BF_n=SG_0+P_n$, with $\deg P_n < \deg G_0, P_n\in\calO[x]$ and $S$ is some quotient. Then,
	\[
		G_0\left(AF_n+H_0Q\right)+H_0P_n\equiv F_n\mod{\pi}
	\]
	
	Omit all coefficients from $AF_n+H_0Q$ divisible by $\pi$ to get $Q_n$.
\end{proof}

\begin{cor-num}
	Let $F(x)=a_0+a_1x+\cdots+a_nx^n\in k[x]$, $k$ complete and $a_0a_n\neq0$. If $F$ is irreducible, then $|a_i|\leq\max(|a_0|,|a_n|)\ \forall i$
\end{cor-num}
\begin{proof}
	WLOG $F$ is primitive (scale). Pick the minimal $r$ s.t. $|a_r|=1$ and look at $F$ modulo $\mathfrak{m}$. If $\max(|a_0|,|a_n|)\neq1$, then $F$ lifts to a non-trivial factorisation via Hensel's Lemma.
\end{proof}

\begin{cor-num}
	$F\in\calO[x]$, $k$ complete. If $F\ \mathrm{mod}\ \mathfrak{m}$ has a simple root $\bar{\alpha}\in k$, then $F$ has a unique simple root $\alpha\in\calO$ lifting $\bar{\alpha}$
\end{cor-num}
\printindex
\end{document}
